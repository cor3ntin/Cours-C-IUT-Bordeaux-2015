\documentclass[8pt]{article}

\usepackage[frenchb]{babel}
\usepackage[latin1]{inputenc}
\usepackage{xltxtra}
%\usepackage{fontspec}
%\setmainfont{Noto Sans}
\usepackage{inconsolata}
\usepackage[a4paper]{geometry}
\usepackage[cm]{fullpage}
\usepackage{listings}
\usepackage{hyperref}
\usepackage[usenames,dvipsnames,svgnames,table]{xcolor}


\definecolor{code}{HTML}{2E2EFE}

\lstset{language=C++,
        escapeinside={(*@}{@*)},
        basicstyle=\ttfamily\color{code},
        showstringspaces=false,
        extendedchars=true,
        breaklines=true,
        showtabs=false,
        showspaces=false,
        showstringspaces=false,
        columns=fullflexible
        } 



\title{\textbf{Introduction à C++11}}
\author{Corentin Jabot}
\date{}

\begin{document}
\NoAutoSpacing


\makeatletter
\begin{center}%
{\LARGE \@title \par}
\end{center}%
\makeatother

\section*{Un peu d'histoire}

La première version de C++ est publiée en 1983. Mais ce n’est qu’en 1998 que C++ est standardisé dans une norme ISO connue sous le nom de \og C++98 \fg{}. En 2011, une version majeure apporte un grand nombre de fonctionnalités qui changent la façon de développer en C++.
La transformation est telle que les auteurs parlent souvent de \og Modern C++ \fg{}.
C++14 généralise certaines fonctionnalités de C++11.

Notez que le commité C++ publie un \og Standard \fg{} (un document), et non un compilateur. Il faut souvent attendre plusieurs années pour que l'ensemble des fonctionnalités soient implémentées par les différents compliateurs C++ existants. Toutefois, Clang, GCC, et MSVC implémentent la majorité du standard C++11, y compris l'ensemble des fonctionnalités présentées en TD et dans ce document.


\section*{auto}

\lstinline{auto} est un mot clé utilisé en lieu et place d'un type, qui laisse la charge au compilateur de déterminer le type de la variable, en fonction de l’expression qui lui est affectée. On parle d'\textbf{inférence de type}.

\begin{lstlisting}
	auto a = 42; // a est de type int
	auto b = "geronimo"; // b est de type const char*
	a = b;  //erreur : impossible de converit un const char* en int
\end{lstlisting}

\lstinline{auto} permet d'éviter les erreurs de conversion implicite, par exemple:
\begin{lstlisting}
	std::vector<Stuff> thingies;
	auto size = thingies.size(); // size est du même type que std::vector::size(), à savoir... std::size_t !
	int incorrectSize =  thingies.size(); //conversion std::size_t -> int
\end{lstlisting}

Et simplifie l'écriture et la lecture du code. Considérez ces 2 façons de parcourir un vecteur, sans et avec \lstinline{auto}:

\begin{lstlisting}
	for(std::vector<std::string>::iterator it = std::begin(vect); it != std::end(vect); ++it )
		std::cout << *it << std::endl;
\end{lstlisting}

\begin{lstlisting}
	for(auto it = std::begin(vect); it != std::end(vect); ++ it)
		std::cout << *it << std::endl;

\end{lstlisting}
		
\section*{Boucle for simplifiée pour les intervalles}
Il existe une façon encore plus simple de parcourir les tableaux et les conteneurs:

\begin{lstlisting}
	std::vector<std::string> vect;
	//[...]
	for(const auto & elem : vect)
		std::cout << elem;
    }
\end{lstlisting}
      
\section*{nullptr}     

En C++, la macro \lstinline{NULL} correspond à la valeur entière \lstinline{0}.

La fonction foo offre une surcharge pour \lstinline{int} et \lstinline{void*},
\begin{lstlisting}
	void foo(int i) {
		std::count << i << "est un int";
	}
	void foo(void* ptr) {
		std::count <<  ptr << "est un pointeur";
	}
\end{lstlisting}

Que pensez-vous qu'affiche \lstinline{foo(NULL);} ? "0 est un int" ... sans doute pas le résultat que vous cherchiez à obtenir !

Pour éviter ce problème il est recommendé de toujours utiliser \lstinline{nullptr} à la place de \lstinline{NULL} ouc' de \lstinline{0} lorsque vous voulez représenter un pointeur nul.  ex : \lstinline{foo(nullptr)}:


\section*{Initialisation uniforme et liste d’initialisation}

C++11 offre une nouvelle syntaxe pour initialiser les variables.

Les déclarations suivantes sont équivalentes
\begin{lstlisting}
	int x{42};
	int x = {42};
	int x(42);
	int x = 42;
\end{lstlisting}

Cette syntaxe peut être utilisée pour initialiser des instances de classes, même lorsque la classe n’as pas de constructeur:

\begin{lstlisting}
	struct Truc {
		int a;
		std::string b;
	};
	Truc t{42, “universe”}; // l'ordre doit respecter l'ordre de déclaration des variables membres de Truc
\end{lstlisting}

Cette syntaxe permet également de remplir des tableaux, des conteneurs, ainsi que d'autres types de la STL

\begin{lstlisting}
    char* alphabet[]  = {"alpha", "beta", "charlie", "delta", nullptr}; //déjà possible en C !
	std::vector<int> numbers = { 4, 8, 15, 16, 23, 42};
	std::map<std::string,std::string> secretIdentities = { 
		{"Bruce Wayne", "Batman"}, 
		{"Matt Murdock", "Daredevil"}
    };
\end{lstlisting}

\section*{Création d'alias de type}

le mot clé \lstinline{using} peut être utilisé pour créer des alias de types afin de simplifier la lecture du code, ou de mieux montrer votre intention

\begin{lstlisting}
	using Username = std::string;
	using Dictionary     = std::map<std::string,std::string>;
	
	Username myUserName{"toto"};
\end{lstlisting}

\section*{Aller plus loin}

C++11 et C++14 offrent bien d'autres fonctionnalités. Lambdas, mots clés "final" et "override" pour l'héritage de classes, templates varidiques, héritage de construction, et la liste continue.

Il vous appartient d'approfondir vos connaissances dans les technologies qui vous interessent.
A cette fin, vous pouvez consulter les sites et ouvrages suivants.

\begin{itemize}
\item \url{http://fr.cppreference.com/w/cpp/language}
\item \url{http://fr.wikipedia.org/wiki/C%2B%2B11}
\item \url{https://isocpp.org/get-started} ( en anglais )
\end{itemize}



\bibliographystyle{plain}
\nocite{*}
\bibliography{references}
\end{document}